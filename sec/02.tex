\section*{Das Inscrições}

\begin{article}
  Cada instituição de ensino poderá inscrever até 10 alunos por nível na
  \currentEdition{} \acrshort{omeg}, além de seus estudantes que tenham sido
  contemplados com medalhas de bronze, prata ou ouro na \previousEdition{}
  \acrshort{omeg}.
\end{article}

\begin{article}
  As inscrições serão realizadas entre os dias \registrationOpening{} e
  \registrationClosing{} inclusive, e devem ser feitas unicamente por meio de
  formulário eletrônico disponível na página da \acrshort{omeg}:
  \begin{center}
    \homepage
  \end{center}
  \begin{description}
    \item[§ 1º]
      A lista com as inscrições homologadas será publicada na página da
      \acrshort{omeg}:
      \begin{center}
        \homepage
      \end{center}
      em \publicationOfAprovedRegistrations{}.
    \item[§ 2º]
      O prazo para interposição de recursos relativos às inscrições é de cinco
      dias úteis a contar da data de publicação da lista de inscrições
      homologadas.
  \end{description}
\end{article}

\begin{article}
  O responsável pelas inscrições de cada escola deverá informar em formulário
  fornecido pela comissão organizadora, os dados seguintes:

  \begin{table}[H]
    \centering
    \rowcolors{1}{white}{gray!10}
    \begin{tabular}{l|l}
      \textbf{Instituições de Ensino} & \textbf{Estudantes}                \\ \hline
      Nome da Instituição             & Nome completo e data de nascimento \\
      Código \acrshort{inep}          & \acrshort{cpf}                     \\
      Endereço completo               & E-mail                             \\
      Telefone para contato           & Nível da prova
    \end{tabular}
  \end{table}

  \begin{description}
    \item[§ 1º]
      Não serão homologadas aquelas inscrições que porventura contenham dados
      incompletos, incorretos ou, no caso de \acrshort{cpf}'s e e-mails,
      duplicados.
    \item[§ 2º]
      Estudantes que não possuam \acrshort{cpf} próprio poderão se utilizar do
      \acrshort{cpf} de um de seus responsáveis.
    \item[§ 3º]
      Caso o estudante necessite de condições especiais para a realização das
      provas, tais necessidades deverão ser informadas no ato de sua inscrição
      na \currentEdition{} \acrshort{omeg}.
  \end{description}
\end{article}

\begin{article}
  O aluno deve verificar seu acesso à plataforma \acrshort{moodle} no período
  de \testingAccessToThePlatformOpeningDate{} a
  \testingAccessToThePlatformClosingDate, conforme instruções disponibilizadas
  em: \homepage. É impreterível que no primeiro acesso, o aluno atualize o
  perfil com uma foto recente e com o nome completo.
  \begin{description}
    \item[§ 1º]
      Caso o aluno não consiga realizar o acesso à plataforma ou deseje
      solicitar a correção de algum dado, a solicitação deve ser feita no
      período de \fixAccessCredentialsOpeningDate{} a
      \fixAccessCredentialsClosingDate{} pelo e-mail \contactUs.
    \item[§ 2º]
      Solicitações de acesso à plataforma \acrshort{moodle} ou correções
      enviadas após o dia \fixAccessCredentialsClosingDate{} não serão
      consideradas e o aluno poderá ter sua inscrição cancelada.
  \end{description}
\end{article}

\begin{article}
  Caso o aluno tenha alguma necessidade especial para realização da prova,
  esta deve ser informada no ato de inscrição, via campo específico no
  formulário de inscrição.
\end{article}
