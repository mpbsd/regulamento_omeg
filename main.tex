\documentclass[a4paper,12pt]{article}

\usepackage{amsmath,amsthm}
\usepackage{float}
\usepackage[acronym]{glossaries}
\usepackage{hyperref}
\usepackage[table]{xcolor}

\newacronym{ufg}{UFG}{Universidade Federal de Goiás}
\newacronym{ime}{IME}{Instituto de Matemática e Estatística}
\newacronym{face}{FACE}{Faculdade de Administração, Contabilidade e Economia}
\newacronym{omeg}{OMEG}{Olimpíada de Matemática do Estado de Goiás}
\newacronym{cpf}{CPF}{Cadastro de Pessoa Física}
\newacronym{moodle}{Moodle}{Modular Object-Oriented Dynamic Learning Environment}
\newacronym{inep}{INEP}{Instituto Nacional de Estudos e Pesquisas Educacionais Anísio Teixeira}

\newtheorem{article}{Art.}

\def\url{https://omeg.ime.ufg.br}
\def\homepage{\href{\url}{\texttt{\url}}}

\def\emailaddres{omeg.ime@ufg.br}
\def\contactUs{\href{mailto:\emailaddres}{\texttt{\emailaddres}}}

\def\moodleufg{https://pesquisaextensao.ead.ufg.br}

\def\previousEdition{XXXI}
\def\currentEdition{XXXII}

\def\year{2023}

\def\registrationOpening{01 de agosto de \year}
\def\registrationClosing{20 de agosto de \year}

\def\publicationOfAprovedRegistrations{05 de setembro de \year}

\def\testingAccessToThePlatformOpeningDate{23 de outubro de \year}
\def\testingAccessToThePlatformClosingDate{25 de outubro de \year}

\def\fixAccessCredentialsOpeningDate{23 de outubro de \year}
\def\fixAccessCredentialsClosingDate{25 de outubro de \year}

\def\phaseOne{28 de outubro de \year}
\def\phaseTwo{11 de novembro de \year}

\def\remoteBasesDisclosureDate{10 de novembro de \year}

\def\resultsFromPhaseOne{31 de outubro de \year}
\def\resultsFromPhaseTwo{29 de janeiro de 2024}

\def\prizesDay{31 de janeiro de 2024}

\begin{document}

\begin{center}
  \texttt{Universidade Federal de Goiás}                    \\
  \texttt{Instituto de Matemática e Estatistica}            \\
  \texttt{XXXII Olimpíada de Matemática do Estado de Goiás}
\end{center}

\vspace{24pt}

A comissão organizadora da \currentEdition{} \acrfull{omeg} torna público e faz
saber através deste documento as instruções e informações referentes à
realização do evento em \year.

\vspace{24pt}

\begin{article}
  A \acrshort{omeg} se destina a estudantes que estejam cursando os Ensinos
  Fundamental ou Médio em escolas públicas ou particulares no Estado de Goiás
  em \year.
\end{article}

\begin{article}
  Os estudantes serão divididos em três níveis, a saber:
  \begin{table}[H]
    \centering
    \rowcolors{1}{white}{gray!10}
    \begin{tabular}{ll}
      \textbf{Nível 1} & 6° e 7° anos do Ensino Fundamental \\
      \textbf{Nível 2} & 8° e 9° anos do Ensino Fundamental \\
      \textbf{Nível 3} & Ensino Médio.
    \end{tabular}
  \end{table}
\end{article}

\section*{Das Inscrições}

\begin{article}
  Cada instituição de ensino poderá inscrever até 10 alunos por nível na
  \currentEdition{} \acrshort{omeg}, além de seus estudantes que tenham sido
  contemplados com medalhas de bronze, prata ou ouro na \previousEdition{}
  \acrshort{omeg}.
\end{article}

\begin{article}
  As inscrições serão realizadas entre os dias \registrationOpening{} e
  \registrationClosing{} inclusive, e devem ser feitas unicamente por meio de
  formulário eletrônico disponível na página da \acrshort{omeg}:
  \begin{center}
    \homepage
  \end{center}
  \begin{description}
    \item[§ 1º]
      A lista com as inscrições homologadas será publicada na página da
      \acrshort{omeg}:
      \begin{center}
        \homepage
      \end{center}
      em \publicationOfAprovedRegistrations{}.
    \item[§ 2º]
      O prazo para interposição de recursos relativos às inscrições é de cinco
      dias úteis a contar da data de publicação da lista de inscrições
      homologadas.
  \end{description}
\end{article}

\begin{article}
  O responsável pelas inscrições de cada escola deverá informar em formulário
  fornecido pela comissão organizadora, os dados seguintes:

  \begin{table}[H]
    \centering
    \rowcolors{1}{white}{gray!10}
    \begin{tabular}{l|l}
      \textbf{Instituições de Ensino} & \textbf{Estudantes}                \\ \hline
      Nome da Instituição             & Nome completo e data de nascimento \\
      Código \acrshort{inep}          & \acrshort{cpf}                     \\
      Endereço completo               & E-mail                             \\
      Telefone para contato           & Nível da prova
    \end{tabular}
  \end{table}

  \begin{description}
    \item[§ 1º]
      Não serão homologadas aquelas inscrições que porventura contenham dados
      incompletos, incorretos ou, no caso de \acrshort{cpf}'s e e-mails,
      duplicados.
    \item[§ 2º]
      Estudantes que não possuam \acrshort{cpf} próprio poderão se utilizar do
      \acrshort{cpf} de um de seus responsáveis.
    \item[§ 3º]
      Caso o estudante necessite de condições especiais para a realização das
      provas, tais necessidades deverão ser informadas no ato de sua inscrição
      na \currentEdition{} \acrshort{omeg}.
  \end{description}
\end{article}

\begin{article}
  O aluno deve verificar seu acesso à plataforma \acrshort{moodle} no período
  de \testingAccessToThePlatformOpeningDate{} a
  \testingAccessToThePlatformClosingDate, conforme instruções disponibilizadas
  em: \homepage. É impreterível que no primeiro acesso, o aluno atualize o
  perfil com uma foto recente e com o nome completo.
  \begin{description}
    \item[§ 1º]
      Caso o aluno não consiga realizar o acesso à plataforma ou deseje
      solicitar a correção de algum dado, a solicitação deve ser feita no
      período de \fixAccessCredentialsOpeningDate{} a
      \fixAccessCredentialsClosingDate{} pelo e-mail \contactUs.
    \item[§ 2º]
      Solicitações de acesso à plataforma \acrshort{moodle} ou correções
      enviadas após o dia \fixAccessCredentialsClosingDate{} não serão
      consideradas e o aluno poderá ter sua inscrição cancelada.
  \end{description}
\end{article}

\begin{article}
  Caso o aluno tenha alguma necessidade especial para realização da prova,
  esta deve ser informada no ato de inscrição, via campo específico no
  formulário de inscrição.
\end{article}

\section*{Das Provas}

\begin{article}
  A \currentEdition{} \acrshort{omeg} será realizada em duas etapas: remota,
  através da Internet, e presencial, nas dependências da \acrshort{ufg}.
\end{article}

\begin{article}
  A primeira etapa será realizada de maneira remota pelo \acrshort{moodle} Ipê
  da \acrshort{ufg}:
  \begin{center}
    \href{\moodleufg}{\moodleufg}
  \end{center}
  dia \phaseOne, das 14:00 às 17:00 horas.
  \begin{description}
    \item[§ 1º]
      Esta etapa consistirá de 10 questões de múltipla escolha, cada uma com
      valor de 1 ponto, e sua correção será feita, automaticamente, pela
      plataforma \acrshort{moodle} Ipê da \acrshort{ufg}.
    \item[§ 2º]
      No dia da prova será aberta uma sala virtual para identificação e
      monitoramento dos alunos durante o período de realização da prova. O
      aluno que não estiver presente na sala virtual no horário de início da
      prova, será considerado desclassificado.
    \item[§ 3º]
      O link para acessar a sala será disponibilizado em:
      \begin{center}
        \homepage
      \end{center}
    \item[§ 4º]
      O fiscal responsável pela sala fará a primeira chamada às 13:50 e a
      última chamada às 14:10. O aluno deve estar em posse de um documento de
      identificação durante toda a realização da prova. Caso o aluno não
      responda a pelo menos uma das chamadas, estará desclassificado.
    \item[§ 5º]
      Os resultados serão divulgados na página
      \begin{center}
        \homepage
      \end{center}
      em \resultsFromPhaseOne.
    \item[§ 6º]
      Os alunos serão classificados em ordem decrescente de nota, e será
      estabelecida uma nota de corte para cada nível, que será disponibilizada
      juntamente com a lista de classificação.
    \item[§ 7º]
      Serão convidados a participar da segunda etapa todos os alunos, de cada
      nível, com notas maiores ou iguais à nota de corte.
    \item[§ 8º]
      A lista de alunos convidados para segunda fase será disponibilizada na
      página \homepage constando apenas os nomes em ordem
      alfabética.
  \end{description}
\end{article}

\begin{article}
  A segunda etapa será realizada no dia \phaseTwo, das 14:00 às 18:00 horas
  de forma presencial no campus Samambaia da \acrshort{ufg} e nos pólos de
  aplicação virtuais divulgados até dia \remoteBasesDisclosureDate{} via site
  da \acrshort{omeg}.
  \begin{description}
    \item[§ 1º]
      Esta etapa consistirá de quatro questões dissertativas, cada uma com
      valor de 10 pontos, sendo que pontuações parciais poderão ser
      consideradas.
    \item[§ 2º]
      Os alunos selecionados para a segunda etapa deverão comparecer ao
      local de prova obrigatoriamente munidos de documento de identificação
      original com foto.
    \item[§ 3º]
      Na segunda etapa, a comissão divulgará os locais de realização da prova
      através de comunicado na página
      \begin{center}
        \homepage
      \end{center}
    \item[§ 4º]
      Os alunos de cada nível serão classificados em ordem decrescente de
      nota para estabelecer as premiações.

\section*{Da Premiação}

    \item[§ 5º]
      O resultado final com a lista de medalhistas e menções honrosas será
      divulgado na página da \acrshort{omeg}:
      \begin{center}
        \homepage
      \end{center}
      em \resultsFromPhaseTwo.
    \item[§ 6º]
      A cerimônia de premiação será realizada dia \prizesDay{} às 09:00 no
      auditório da Biblioteca Central localizado no campus Samambaia da UFG.
  \end{description}
\end{article}

\section*{Disposições Finais}

\begin{article}
  Qualquer fato novo, omissão, dúvida ou caso fortuito oriundo do presente
  regulamento será decidido pela comissão organizadora.
\end{article}

\begin{article}
  Ao aceitar participar da \currentEdition{} \acrshort{omeg}, os
  estabelecimentos de ensino, os responsáveis pelos alunos menores de idade e
  os alunos maiores de idade aceitam o presente regulamento e as condições em
  questão.
\end{article}

Goiânia, 28 de julho de \year.

\section*{Comissão Organizadora}

\begin{table}[H]
  \centering
  \rowcolors{1}{white}{gray!10}
  \begin{tabular}{llll}
    Profa. & Dra. & Rosane Gomes Pereira\footnote{coordenadora da comissão organizadora}                  & \acrshort{ime}  \\
    Prof.  & Dr.  & Otávio Marçal Leandro Gomide\footnote{vice-coordenador da comissão organizadora}      & \acrshort{ime}  \\
    Prof.  & Dr.  & Tiago Moreira Vargas\footnote{presidente comissão de elaboração e correção de provas} & \acrshort{ime}  \\
    Prof.  & Dr.  & Marcelo Bezerra Barboza\footnote{coordenador de inscrições}                           & \acrshort{ime}  \\
    Prof.  & Dra. & Adriana Araujo Cintra\footnote{coordenadora de divulgação}                            & \acrshort{ime}  \\
    Profa. & Dra. & Ana Paula de Araújo Chaves                                                            & \acrshort{ime}  \\
    Prof.  & Dr.  & Francisco Bruno de Lima Holanda                                                       & \acrshort{face} \\
    Prof.  & Dr.  & Gregory Duran Cunha                                                                   & \acrshort{ime}  \\
    Profa. & Dra. & Kamila da Silva Andrade                                                               & \acrshort{ime}  \\
    Prof.  & Dr.  & Luiz Fernando Gonçalves                                                               & \acrshort{ime}  \\
    Prof.  & Dr.  & Valdivino Vargas Júnior                                                               & \acrshort{ime}  \\
    TAE    &      & Mislene da Silva Gomes                                                                & \acrshort{ime}
  \end{tabular}
\end{table}

\end{document}
