                                       UNIVERSIDADEFEDERALDEGOIÁS
                                  INSTITUTODEMATEMÁTICAEESTATÍSTICA
                          XXXII OLIMPÍADADEMATEMÁTICADOESTADODEGOIÁS
                                                   REGULAMENTO
               A comissão organizadora da XXXII Olimpíada de Matemática do Estado de Goiás
               (OMEG) torna públicas e faz saber, por meio deste documento, as instruções e
               informações referentes à realização do evento em 2023.
                 1.   A OMEG se destina aos alunos regularmente matriculados e cursando o
                      Ensino Fundamental ou Ensino Médio em 2023 em escolas públicas ou
                      particulares do Estado de Goiás, os quais serão divididos em três níveis, a
                      saber:
                      Nível 1: alunos do 6° e 7° anos do Ensino Fundamental I;
                      Nível 2: alunos do 8° e 9° anos do Ensino Fundamental II;
                      Nível 3: alunos do Ensino Médio.
                                                  DASINSCRIÇÕES
                 2.   Cada estabelecimento de ensino poderá inscrever na XXXII OMEG, além dos
                      premiados com medalha de ouro, prata ou bronze na XXXI OMEG, até 10
                      alunos por nível.
                 3.   As inscrições serão realizadas no período de 01/08/2023 a 20/08/2023,
                      exclusivamente      por   meio    de    formulário   eletrônico    disponível   em
                      https://omeg.ime.ufg.br/.
                      3.1.   A lista     de inscrições     homologadas será publicada no site
                      https://omeg.ime.ufg.br/ no dia 05 de setembro de 2023 e o prazo para
                      interposição de recursos será de cinco dias úteis.
                 4.   O responsável pela inscrição deverá informar em formulário específico
                      fornecido pela comissão os seguintes dados:
                          ● Estabelecimento de Ensino: nome da escola, código INEP, endereço
                             completo e número de telefone.
                          ● Aluno: nome completo, data de nascimento, CPF, e-mail de contato
                             individual e nível da prova que o aluno realizará, e, caso o aluno
                             necessite de condições especiais para realizar as provas, a descrição
                             detal necessidade.
                      4.1 As inscrições de alunos que contenham dados incompletos, incorretos ou
                      duplicados não serão realizadas.
                      4.2 Caso o aluno não possua CPF é permitido o uso do CPF de um de seus
                      responsáveis.
                 5.   O aluno deve verificar seu acesso à plataforma Moodle no período de
                      16/10/2023     a 23/10/2023, conforme instruções disponibilizadas em:
                      https://omeg.ime.ufg.br/ . É impreterível que no primeiro acesso, o aluno
                      atualize o perfil com uma foto recente e com o nome completo.
                      5.1 Caso o aluno não consiga realizar o acesso à plataforma ou deseje
                      solicitar a correção de algum dado, a solicitação deve ser feita no período de
                      23/10/2023 a 25/10/2023 pelo e-mail omeg.ime@ufg.br.
                      5.1.1 Solicitações de acesso à plataforma Moodle ou correções enviadas
                      após o dia 25/10/2023 não serão consideradas e o aluno poderá ter sua
                      inscrição cancelada.
                 6.   Caso o aluno tenha alguma necessidade especial para realização da prova,
                      esta deve ser informada no ato de inscrição, via campo específico no
                      formulário de inscrição.
                                                     DASPROVAS
                 7.   AXXXIIOMEGserárealizadaemduasetapas:onlineepresencial.
                 8.   A primeira etapa será realizada online via plataforma Moodle Ipê UFG
                      (https://ead.ufg.br/) dia 28/10/2023, das 14:00 às 17:00 horas.
                      8.1 Esta etapa consistirá de 10 questões de múltipla escolha, cada uma com
                      valor de 1 ponto, e sua correção será feita, automaticamente, pela plataforma
                      Moodle Ipê UFG.
                      8.2 No dia da prova será aberta uma sala virtual para identificação e
                      monitoramento dos alunos durante o período de realização da prova. O
                      aluno que não estiver presente na sala virtual no horário de início da
                      prova, será considerado desclassificado.
                      8.2.1   O link para acessar a sala será disponibilizado na página
                      https://omeg.ime.ufg.br/.
                      8.2.2 O fiscal responsável pela sala fará a primeira chamada às 13:50 e a
                      última chamada às 14:10. O aluno deve estar em posse de um documento de
                      identificação durante toda a realização da prova. Caso o aluno não responda
                      anenhumadaschamadas,eleseráconsideradodesclassificado.
                      8.3 Os resultados serão divulgados na página https://omeg.ime.ufg.br/, no dia
                      31/10/2023.
                      8.4 Os alunos serão classificados em ordem decrescente de nota, e será
                      estabelecida uma nota de corte para cada nível, que será disponibilizada
                      juntamente com a lista de classificação.
                      8.5 Serão convidados a participar da segunda etapa todos os alunos, de cada
                      nível, com notas maiores ou iguais à nota de corte.
                      8.5.1 A lista de alunos convidados para segunda fase será disponibilizada na
                      página https://omeg.ime.ufg.br/ constando apenas os nomes em ordem
                      alfabética.
                 9.   A segundaetapaserárealizada dia 11/11/2023, das 14:00 às 18:00 horas de
                      forma presencial no campus Samambaia da UFG e nos pólos de aplicação
                     virtuais divulgados até dia 10/11/2023 via site da OMEG.
                     9.1 Esta etapa consistirá de quatro questões dissertativas, cada uma com
                     valor   de 10 pontos, sendo que pontuações parciais poderão ser
                     consideradas.
                     9.2 Os alunos selecionados para a segunda etapa deverão comparecer ao
                     local de prova obrigatoriamente munidos de documento de identificação
                     original com foto.
                     9.3 Na segunda etapa, a comissão organizadora da OMEG divulgará os
                     locais   de realização da prova através de comunicado na página
                     (https://omeg.ime.ufg.br)
                     9.4 Os alunos de cada nível serão classificados em ordem decrescente de
                     nota para estabelecer as premiações.
                                                 DAPREMIAÇÃO
                     10.2 O resultado final com a lista de medalhistas premiados e menções
                     honrosas     será    divulgado    na   data    de    29/01/2024    via   página
                     https://omeg.ime.ufg.br/.
                     10.3 A cerimônia de premiação será realizada dia 31/01/2024 às 09:00 no
                     auditório da Biblioteca Central localizado no campus Samambaia da UFG.
                                              DISPOSIÇÕESFINAIS
               10.   Qualquer fato novo, omissão, dúvida ou caso fortuito oriundo do presente
                     regulamento será decidido pela comissão organizadora.
               11.   Ao aceitar participar da XXXII OMEG, os estabelecimentos de ensino, os
                     responsáveis pelos alunos menores de idade e os alunos maiores de idade
                     aceitam o presente regulamento e as condições em questão.
                                                                       Goiânia, 28 de julho de 2023.
              ComissãoOrganizadora
              Profa. Dra. Rosane Gomes Pereira (coordenadora da comissão organizadora)
              Instituto de Matemática e Estatística – UFG
       Prof. Dr. Otávio Marçal Leandro Gomide (vice-coordenador da comissão
       organizadora)
       Instituto de Matemática e Estatística – UFG
       Prof. Dr. Tiago Moreira Vargas (presidente comissão de elaboração e correção de
       provas)
       Instituto de Matemática e Estatística – UFG
       Prof. Dr. Marcelo Bezerra Barboza (coordenador de inscrições)
       Instituto de Matemática e Estatística – UFG
       Prof. Dra. Adriana Araujo Cintra (coordenadora de divulgação)
       Instituto de Matemática e Estatística – UFG
       Profa. Dra. Ana Paula de Araújo Chaves
       Instituto de Matemática e Estatística – UFG
       Prof. Dr. Francisco Bruno de Lima Holanda
       Faculdade de Administração, Contabilidade e Economia – UFG
       Prof. Dr. Gregory Duran Cunha
       Instituto de Matemática e Estatística – UFG
       Profa. Dra. Kamila da Silva Andrade
       Instituto de Matemática e Estatística – UFG
       Prof. Dr. Luiz Fernando Gonçalves
       Instituto de Matemática e Estatística – UFG
       Prof. Dr. Valdivino Vargas Júnior
       Instituto de Matemática e Estatística – UFG
       Mislene da Silva Gomes
       Técnica-Administrativa – UFG
